\documentclass[11pt]{article}
\usepackage[margin = 1.2in, nohead, nofoot]{geometry}
\usepackage{color}
\usepackage{hyperref}
\title{How to use the Anderson-Moore Algorithm with EViews and R}
\author{Aneesh Raghunandan}
\newcommand{\gc}[1]{{\color{blue} #1}}
\date{\today} 
\begin{document}
\maketitle
\section{Using EViews} \gc{Looks pretty good!!.  It will be very helpful.}
\subsection{Preliminaries} 
Before you begin, you will need the ``R and Friends'' software, found \href{http://rcom.univie.ac.at/download.html}{here}.  This is the software that enables EViews to connect to and use R, utilizing Microsoft's COM technology.  Additionally, since the R package has not been tested on R versions lower than 2.11.0, it would be wise to have at least version 2.11.0 installed on your machine (for 32-bit Windows). \\ \gc{ I had to leave before actually getting EViews and R working together with AMA.  Does one load library in R once and for all or each time R is started.  Does it matter whether Eviews or R gets started first?}
 
The EViews commands you will need to use from the EViews command line are ``xopen''\gc{perhaps italics or bold fonts would  remove the slight ambiguity about using quotes for describing commands}, ``xclose'', ``xrun'', ``xput'', and ``xget''.  The first opens the connection to R or MATLAB; the second closes this connection; the third runs a command in that connection; the fourth places an object into R or MATLAB; and the fifth brings an existing object in the R/MATLAB workspace into EViews. \\

To install the R package, download AMA$\_$1.0.zip.  Open R.  Under the ``packages'' menu, select ``install package from local zip file'', find wherever you saved the AMA$\_$1.0.zip file, and select it.  The package should install correctly.  To test this, type ``library(AMA)'' at the R command line.  If no error message appears, the package has loaded successfully. \\ \gc{{ You might emphasize how important it is to ignore the first item in the list and to selelct the last item in the drop down list.  I made that mistake and I had trouble selecting a CRAN repository. It took about 3-4minutes to produce the list of repositories and then the I got the warning, "unable to connect" followed by an hourglass for another 5-8 minutes.}}

\subsection{Model Processing}
Documentation and an EViews subroutine to convert an EViews Model object to the structural matrices required by the Anderson-Moore Algorithm are forthcoming.  In the mean time, there are two ways to get the model into a form usable by AMA:
\begin{enumerate}
\item Structural matrices (the $H$ matrices) already in EViews
\item Model in MODELEZ syntax in separate file (for details of MODELEZ syntax, see below)
\end{enumerate}
In the first case, the procedure is straightforward and will be explained below.  In the second case, you will be able to import the structural matrices to EViews (if you like); regardless, you can execute all commands from within EViews. \\

\subsubsection{The Structural Coefficients Matrix H Needed by the Algorithm}
The Anderson-Moore Algorithm defines and uses a matrix of structural coefficients for each lag and lead (and present state) in the model.  If the model has $\tau$ lags and $\theta$ leads, then we have the matrices $H_{-\tau},...,H_0, ...,H_{\theta}$.  The ``structural coefficients'' matrix needed by the algorithm is simply the block matrix $\left[H_{-\tau}\cdots H_0 \cdots H_{\theta} \right] $.  These can be computed by providing the model equations and parameters (which are perhaps initialized by data or simulation).

\subsubsection{MODELEZ Syntax}
If you want to use this format, save the model in a file ending with .mod\gc{if using mod causes problems we could insist on another extension ( how about ama. unlikely windows wants that.)  we could even leave of the extension if windows allows it.} (it may be necessary to save it as a .txt file first then manually change the file extension) using the methods and syntax described below.  The example in the last section includes a sample MODELEZ file. \\
The first line of the file should be the model name, written as follows \begin{itemize}
\item MODEL $>$ NAMEOFMODEL 
\end{itemize}
Next, list the variables (note: in the AMA formulation, all variables are considered endogenous) as follows: \begin{itemize}
\item ENDOG $>$ \\ $\makebox[0.3in][1]{}$ variable1 \\ $\makebox[0.3in][1]{}$ variable2 \\ $\makebox[0.45in][1]{} \vdots$ 
\end{itemize}
After this, you'll need to list the equations.  The two possible values for EQTYPE are IMPOSED and STOCH.\gc{Despite what other documention implies, 
sparseAMA does not use EQTYPE.  I'd rather this were listed as an optional feature available for use  by other routines.}  Write each equation as follows: \begin{itemize}
\item EQUATION $>$ NAMEOFEQUATION \\ EQTYPE $>$ TYPE (either IMPOSED or STOCH) \\
EQ $>$ (model equation--see below)
\end{itemize}
The model equation should be written in the general form dependent variable = ...; where leads and lags are appropriate, use LEAD(variablename, numberOfLead) or LAG(variablename, numberOfLag).\gc{The parser accepts both LHS=RHS and EQN alone so that EQN=0.  sparseAMA does not distinguish between dependent and independent variables and the LHS could be an expression.}  For example, if your model includes the equation $Y_t = \delta Y_{t-2} + \beta IS_{t+3} + K$ you would write the equation in MODELEZ as (assuming DELTA corresponds to $\delta$ and BETA corresponds to $\beta$): \\
\begin{center} Y = DELTA*LAG(Y, 2) + BETA*LEAD(IS, 3) + K  \end{center}

After inputting all variables and equations into your .mod file, the last line of the file should just be ``END'' (without the quotation marks, and in all CAPS). \\

The example MODELEZ file will hopefully make this clearer. \\

\subsection{Computing using R}
Note: for details about the arguments to each function, please see the ``AMA-Manual.pdf'' file.  It is written in the R documentation standard.  For each function you will find a list of inputs as well as a description of each input, the function's output, and in some cases a short example of how to use the function. \\

Before carrying out any computations, you need to open the connection to R and load the AMA library.  Run the following: \begin{itemize}
\item xopen(R) // (if you haven't already)
\item xrun (library(AMA)) // this loads the AMA package\gc{I had to leave before I got this to work. I'll check it out tomorrow.}
\end{itemize}
Note: you need to have the `RJava' package installed. This comes automatically with almost all R distributions, however, and so you almost certainly already have it.
\subsubsection{Getting the $H$ matrix into R}
If you have the structural coefficients matrix $H$ already in the EViews workfile, then to place this matrix into \bfseries R \normalfont you would do the following: \begin{itemize}
\item Run the command `xput(hmat)' (where hmat is the name I've chosen to give the matrix $H$--you can replace this with whatever name you'd like) 
\end{itemize}

If you do not have the structural coefficients matrix $H$ already in the EViews workfile, but you have it in a MODELEZ file (say, myModel.mod) then to get the H matrix into R you would do the following, utilizing the AMA package's function genHmat: \begin{itemize}
\item If you have the model parameters stored in a text file (say, myParams.txt), then run the command `xrun (hmat $<$- genHmat(``myModel.mod'', ``myParams.txt'') )'. 
\item  If you instead have the model parameters stored in an EViews vector object, say ``myParamVec'', then you should first run ``xput myParamVec'' and then run the command `xrun(hmat $<$- genHmat(``myModel.mod'', myParamVec))'.
\item If you have the model parameters stored in a text file (say, myParams.txt) AND would like to obtain not only the matrix H but the parameters in a vector as well (in order to easily update them in EViews or R), the procedure is slightly more complicated.  Instead of the previous two options, you would run the following commands (after opening the connection to R):
\begin{itemize}
  \item xrun(hlist $<$- genHmat(``myModel.mod'', ``myParams.txt'', wantParamVec = TRUE))
  \item Extract the H matrix using the command `xrun(hmat$<$- hlist[1][[1]])', and the parameter vector using the command `xrun(myParamVec $<$- hlist[2][[1]])' 
\end{itemize} %inner itemize
\end{itemize}
NOTE: Wherever you see ``myModel.mod'' or ``myParams.txt'' above, replace these with the full path to the respective files (for example, ``C:/Users/Aneesh/Documents/AMA/myModel.mod'').  Remember to use the quotation marks! Otherwise, R won't recognize this as a string object.

\subsubsection{Next Steps}
You should now have the $H$ matrix into R.  At the moment, however, you can't do anything with it! 

You can now compute any of the matrices associated with AMA.  You will need to tell these functions how many equations, lags, and leads are in the model.  To do so: \begin{itemize}
\item xrun (neq = $<$number of equations$>$) 
\item xrun (leads = $<$number of leads in the model$>$)
\item xrun (lags = $<$number of lags in the model$>$) 
\end{itemize}
To compute the matrices you desire, please see AMA-Manual.pdf for a list of functions, their arguments, descriptions of their arguments, and description of their outputs.  Just xrun the function you desire.  For example, if you wanted to compute the reduced-form coefficients matrix $B$, you would type (in EViews) \begin{itemize}
\item xrun(bmat $<$- genBmat(hmat, neq, leads, lags))
\end{itemize}
This creates an object in the R workspace called ``bmat''.  To bring it back into EViews, you would run `xget(bmat)'; this places it in the current EViews workfile as a matrix called ``bmat''. (Note: this means that you should assign distinct names to objects! EViews will often complain if you try to replace an object with something else of the same name). \\

Note that you can generate the matrices $B$ and $Q$ directly via a call to the function ``callAMA''.  However, you'll get them back as vectors (which correspond to matrices stored columnwise) which may create unnecessary hassle if you're trying to bring them back into EViews.  To get around this, use the genBmat, genQmat, etc. functions directly.  You don't need to use callAMA first--these functions call AMA already.  \\
Obviously for functions such as genScof, you'll have to run genBmat first (since genScof takes the matrix $B$ as an input).  Be aware of which matrices depend on which. \\

\subsection{Using R as a standalone}
The idea here is extremely similar to the EViews examples above.  Just remove the ``xput'', ``xget'', and ``xrun'' statements. 

\subsubsection{Linux side}
Although EViews does not run on Linux, if you wish to use the AMA R package on the Linux side, it is a matter of two things: \begin{enumerate} 
\item Having R installed on your Linux machine.  This means that you should be able to issue commands of the form ``R CMD ...'' from the shell.
\item Assuming this is the case, download AMA$\_$1.x.tar.gz (at the time of this writing, x = 0).  From the shell, issue the command ``R CMD INSTALL AMA$\_$1.x.tar.gz'' (replacing $x$ with the appropriate value, of course, and omitting the quotation marks).  The AMA package should now be installed on your Linux machine.  From here, the commands to issue are identical to the Windows side. 
\end{enumerate} 

\section{Examples}
We walk through a simple example from EViews, using a MODELEZ file.  In the ``Examples'' folder, the model in MODELEZ syntax is ``example7'', and the parameter file is ``example7params.txt''.  From the EViews side, we run the following commands with an explanation of what they do: \\

\indent xopen(r) \quad // opens R \\
\indent xrun(library(AMA)) \quad // loads the AMA (and rJava) library \\
\indent xrun(hmat $<$- genHmat(``example7.mod'', ``example7params.txt'') \quad //creates H matrix \\
\indent xget(hmat) \quad //brings H matrix into EViews \\
\indent xrun(neq = 4) \quad  // number of equations \\
\indent xrun(nlag = 1) \quad // number of leads \\
\indent xrun(nlead = 1) \quad // number of lags \\
\indent xrun(bmat $<$- genBmat(hmat, neq, nlag, nlead)) \quad // creates reduced-form coeff. matrix \\
\indent xget(bmat) \quad //imports reduced-form coeff. matrix into EViews \\
\indent xrun(scof $<$- genScof(hmat, bmat, neq, nlag, nlead)) \quad //creates ob. struct.  matrix \\
\indent xget(scof) \quad // imports observable structure matrix into EViews \\
\indent xrun(factMats $<$- getFactorMatrices(hmat, bmat, neq, leads, lags)) \quad // makes $\phi, F$... \\
\indent xrun(phiMat $<$- factMats[1][[1]]) \quad //and gets the matrix $\phi$ explicitly... \\
\indent xrun(fMat $<$- factMats[2][[1]]) \quad // and the matrix $F$ too. \\
\indent xget(phiMat) \quad // imports matrix $\phi$ into EViews \\
\indent xget(fMat) \quad // imports matrix $F$ back into EViews \\
\indent xrun(stochMats $<$-getStochTrans(hmat, scof) ) \quad // stochastic transition matrices...\\
\indent xrun(scriptA $<$- stochMats[1][[1]]) \quad // gets the matrix $\mathcal{A}$ back... \\
\indent xrun(scriptB $<$- stochMats[2][[1]]) \quad // and the matrix $\mathcal{B}$ too. \\
\indent xget(scriptA) \quad // imports matrix $\mathcal{A}$ into EViews \\
\indent xget(scriptB) \quad // imports matrix $\mathcal{B}$ into EViews \\
\indent xclose(r) \quad // closes the connection to R \\

Note that if you reopen the connection to R, all variables must be redefined.  Make sure to use xget on all the variables you want before closing R, and that you've carried out all the computations you wanted to use R to do. \\

When the EViews model-to-matrix parser is available, an example of how to use that parser to generate the $H$ matrix will appear in this section as well.  If the matrix $H$ is already in EViews (and this will be the case if the model-to-matrix parser is used), then instead of using the genHmat function in the example above, just use xput(hmat) instead.  The rest is unchanged.


\end{document}
